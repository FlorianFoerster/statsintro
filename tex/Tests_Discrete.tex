\chapter{Tests on Discrete Data}

Data can be discrete for different reasons. On is that you acquired them in a discrete way (e.g. levels in a questionnaires.) Another one is that your paradigm only gives discrete results (e.g. rolling a dice). For the analysis of such data, we can build on the tools that we have already covered in the previous chapters.

\section{Tests on Ordinal Data}

Ordinal data have clear rankings, e.g. "none - little - some - much - very much". However they are not continuous. For the analysis of such \emph{rank ordered data} we can use rank order methods for the analysis:

\begin{description}
  \item[Two groups] When comparing two rank ordered groups, we can use the \emph{Mann-Whitney test} \ref{test:Mann-Whitney}
  \item[Three or more groups]  When comparing two rank ordered groups, we can use the \emph{Kruskal-Wallis test} \ref{test:Kruskal-Wallis}

\end{description}

\section{Binomial Test}\index{general}{test!binomial}

\subsection{Example}
Suppose we have a board game that depends on the roll of a die and attaches special importance to rolling a 6. In a particular game, the die is rolled 235 times, and 6 comes up 51 times. If the die is fair, we would expect 6 to come up 235/6 = 39.17 times. Is the proportion of 6's significantly higher than would be expected by chance, on the null hypothesis of a fair die?

To find an answer to this question using the \emph{Binomial Test}, we consult the binomial distribution with n=235 and p=1/6, to determine the probability of finding exactly 51 sixes in a sample of 235 if the true probability of rolling a 6 on each trial is 1/6. We then find the probability of finding exactly 52, exactly 53, and so on up to 235, and add all these probabilities together. In this way, we calculate the probability of obtaining the observed result (51 6s) or a more extreme result ($>51 6's$) assuming that the die is fair. In this example, the result is 0.0265, which indicates that observing 51 6's is unlikely (significant at the 5\% level) to come from a die that is not loaded to give many 6's (one-tailed test).

Clearly a die could roll too few sixes as easily as too many and we would be just as suspicious, so we should use the two-tailed test which (for example) splits the 5\% probability across the two tails.

\PyImg "binomial.py" (p \pageref{py:binomial}): Example of a one-and two-sided binomial test.
\index{python}{binomialTest}

\section{Exercises}

\begin{enumerate}
    \item Under which conditions do you use the \emph{binomial distribution} to evaluate the likelihood of a discrete number of events? And under which do you use the \emph{Poisson distribution}?
\end{enumerate}