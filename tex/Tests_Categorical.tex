\chapter{Tests on Categorical Data }

In a sample of individuals the number falling into a particular group is called the \emph{frequency}, so the analysis of categorical data is the analysis of frequencies. When two or more groups are compared the data are often shown in the form of a \emph{frequency table}, sometimes also called \emph{contingency table}.

\begin{table}
  \centering
  \begin{tabular}{|c|l l l|}
  \hline
  & \emph{Right Handed} & \emph{Left Handed} & \emph{Total} \\
  \hline
  \emph{Males} & 43 & 9 & 52 \\
  \emph{Females} & 44 & 4 & 48 \\
  \emph{Total} & 87 & 13 & 100 \\
  \hline
  \end{tabular}

  \caption{Example of a frequency table}\label{table:frequency}
\end{table}


\section{One Proportion }

If you have one sample group of data, you can check if your sample is representative of the standard population. To do so, you have to know the proportion $p$ of the characteristic in the standard population. It can be shown that a in population with a characteristic with probability $p$, the standard error of samples with this characteristic is given by

\begin{equation}
  se(p) = \sqrt{p(1-p)/n}
\end{equation}

and the corresponding 95\% confidence interval is
\begin{equation*}
  ci = mean \pm se * t_{n,0.95}
\end{equation*}

If your data lie outside this confidence interval, they are \emph{not} representative of the population.

\section{Frequency Tables} \index{general}{frequency tables}

\subsection{Chi-square Test} \index{general}{test!chi square}

Assume you have observed absolute frequencies $o_i$ and expected absolute frequencies $e_i$ under the Null hypothesis of your test then it holds

\begin{equation}
  V = \sum_i \frac{(o_i-e_i)^2}{e_i} \approx \chi^2_f
\end{equation}.

where $f$ are the degrees of freedom. $i$ might denote a simple index running from $1,...,I$ or even a multiindex $(i_1,...,i_p)$ running from $(1,...,1)$ to $(I_1,...,I_p)$.

The test statistic $V$ is approximately $\chi^2$ distributed, if

\begin{itemize}
  \item for all absolute expected frequencies $e_i$ holds $e_i \geq 1$ and
  \item for at least 80\% of the absolute expected frequencies $e_i$ holds $e_i \geq 5$.
\end{itemize}

The degrees of freedom can be computed by the numbers of absolute observed frequencies which can be chosen freely. We know that the sum of absolute expected frequencies is

\begin{equation}
  \sum_i o_i = n
\end{equation}

which means that the maximum number of degrees of freedom is $I-1$. We might have to subtract from the number of degrees of freedom the number of parameters we need to estimate from the sample, since this implies further relationships between the observed frequencies.

\subsubsection{Example}

The $\chi^2$ test can be used to generate "quick and dirty" test, e.g.

$H_0:$ The random variable $X$ is symmetrically distributed versus

$H_1:$ the random variable $X$ is not symmetrically distributed.

We know that in case of a symmetrical distribution the arithmetic mean $\bar{x}$ and median should be nearly the same. So a simple way to test this hypothesis would be to count how many observations  are less than the mean ($n_-$)and how many observations are larger than the arithmetic mean ($n_+$). If mean and median are the same than 50\% of the observation should smaller than the mean and 50\% should be larger than the mean. It holds

\begin{equation}
  V = \frac{(n_- - n/2)^2}{n/2} + \frac{(n_+ - n/2)^2}{n/2} \approx \chi^2_1
\end{equation}.


\subsubsection{Comments}
The Chi-square test is a pure hypothesis test. It tells you if your observed frequency can be due to a random sample selection from a single population. A number of different expressions have been used for chi-square tests, which are due to the original derivation of the formulas (from the time before computers were pervasive). Expression such as \emph{2x2 tables}, \emph{r-c tables}, or \emph{Chi-square test of contingency} all refer to frequency tables and are typically analyzed with chi-square tests.

\subsection{Fisher's Exact Test} \index{general}{test!Fisher's exact}
For small sample numbers, corrections should be made for some bias that is caused by the use of the continuous chi-squared distribution. This correction is referred to as \emph{Yates correction}.

If the requirement that 80\% of cells should have expected values of at least 5 is not fulfilled, \emph{Fisher's exact test} should be used. This test is based on the observed row and column totals. The method consists of evaluating the probability associated with all possible 2x2 tables which have the same row and column totals as the observed data, making the assumption that the null hypothesis (i.e. that the row and column variables are unrelated) is true.
In most cases, Fisher's exact test is preferable to the chi-square test. But until the advent of powerful computers, it was not practical. You should use it up to approximately 10-15 cells in the frequency tables.

\section{Analysis Programs}

With computers, the computational steps are trivial:

\lstinputlisting[label=py_compGroups,caption=compGroups.py, language=Python]{../Code/compGroups.py}
\index{python}{compGroups}
